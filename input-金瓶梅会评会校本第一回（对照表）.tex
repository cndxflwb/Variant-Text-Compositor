\documentclass[UTF8,scheme=chinese,11pt,linespread=1.4]{ctexbook}

\ctexset{
	tocdepth=section,
	part/numbering=false,
	part/pagestyle=empty,
	chapter/numbering=false,
	chapter/pagestyle=empty,                  %章級標題頁面不顯示頁眉頁腳
	section/numbering=false,                  %節級標題不編號
	subsection/numbering=false,                  %節級標題不編號
}

\usepackage[a4paper,outer=1in,inner=1in,bottom=1in]{geometry}

% 控制默認的列表間距
% https://yuxtech.club/2021/01/18/enumerate/
\usepackage{enumitem}
\setlist{nosep}
\AddEnumerateCounter{\chinese}{\chinese}{} % \begin{enumerate}[label = \chinese*、]

% 尾注-用於校勘記
% 用法:文中\endnote{...} 在要显示尾注的位置输入 \theendnotes
\usepackage{endnotes}
\usepackage{etoolbox}% 每章尾注重新编号
\makeatletter
\def\enoteheading{\subsubsection*{\notesname
  \@mkboth{\MakeUppercase{\notesname}}{\MakeUppercase{\notesname}}}%
  \mbox{}\par\vskip-3\baselineskip\noindent\rule{.5\textwidth}{0pt}\par\vskip\baselineskip}
\makeatother
\renewcommand{\notesname}{校勘记} % 定义尾注的标题名称
%\newcommand{\jiaokanji}{\vspace{1em}\noindent\textbf{校勘记}} % 要随\theendnotes一起出现

\def\enotesize{\normalsize}
\csappto{theendnotes}{\setcounter{endnote}{0}}
%\renewcommand{\theendnote}{\hskip -2pt〔\zhdigits{\arabic{endnote}}〕\hskip -2pt} % 中文编号带六角括号

\usepackage{xcolor}

% 分頁表格
\usepackage{array}
\renewcommand{\arraystretch}{1.1} %单元格邊框距內容距離
%\renewcommand\thetable{\zhnumber{table}} %表格序號爲中文編號
\setlength{\fboxsep}{1pt} %边框与内容的距离
% 新一代表格宏包,中文说明:https://gitee.com/nwafu_nan/tabularray-doc-zh-cn
\usepackage{tabularray}
\NewTblrTheme{fancy}{
	\SetTblrStyle{firsthead}{font=\bfseries}
	\SetTblrStyle{middlehead}{font=\bfseries}
	\SetTblrStyle{lasthead}{font=\bfseries}
	%\SetTblrStyle{firstfoot}{fg=blue2}
	\SetTblrStyle{middlefoot}{\itshape}
	\SetTblrStyle{caption-tag}{red2}
}
\DefTblrTemplate{contfoot-text}{default}{转下页}
\DefTblrTemplate{conthead-text}{default}{(续表)}

% 超链接
% \texorpdfstring{显示内容}{书签内容},当标题中存在命令时,需要保护,防止编译报warning
\usepackage{hyperref}
\hypersetup{
	CJKbookmarks=true,                %開啓CJK書籤
	bookmarksopen=false,              %默認書籤不展開,即摺疊狀態
	colorlinks,                       %超鏈接要有顏色
	linkcolor=black,                 %超鏈接的顏色是黑色,要使用這個配置,就取消註釋
	unicode=true,
	bookmarksnumbered=true,          %書籤編號
}

% 版本注册命令(保留用于其他用途)
\newcommand{\banbenzhuce}[1]{}

% 定义banben命令(只输出底本文字)
\newcommand{\banben}[2][]{#2}

\makeatletter
\newenvironment{diben}[1]
  {\list{}{\listparindent 2em
    \itemindent\listparindent
    \rightmargin 0em  % 将右边距设为 0
    \leftmargin 0em  % 保持左边距
    \parsep \z@ \@plus\p@}%
   \item\relax}
  {\endlist}
\makeatother

\begin{document}

% 注册版本
\banbenzhuce{词本,崇本}

\chapter{第一回}

\begin{diben}
第一回

\banben{西门庆热结十兄弟}[崇本={西门庆热结十弟兄},词本={景阳冈武松打虎}] \banben{武二郎冷遇亲哥嫂}[词本={潘金莲嫌夫卖风月}]

此书单重财色,故卷首一诗,上解悲财,下解悲色。

一部炎凉书,乃开首一诗并无热气。信乎作者注意在下半
部,而看官益当知看下半部也。

“二八佳人”一绝,色也。借色说人,则色的利害比财更甚。
下文“一朝马死”二句,财也;“三杯茶作合”二句,酒也;“三寸
气在”二句,气也。然而酒、气俱串入财、色内讲,故诗亦串入。
小小一诗句,亦章法井井如此,其文章为何如?

开讲处几句话头,乃一百回的主意。一部书总不出此几
句,然却是一起、四大股、四小结股、临了一结,齐齐整整一篇
文字。断落皆详批本文下。

上文一律、一绝、三成语,末复煞四句成语,见得痴人不
悟,作孽于酒色财气中,而天自处高听卑,报应不爽也。是作者
盖深明天道以立言欤?《金刚经》四句,又一部结果的主意也。

尝看西门死后,其败落气象,恰如的的确确的事,亦是天
道不深不浅,恰恰好好该这样报应的。每疑作者非神非鬼,何
以操笔如此?近知作者骗了我也。盖他本是向人情中讨出来
的天理,故真是天理。然则不在人情中讨出来的天理,又何以【1】
为之天理战!自家作文,固当平心静气,向人情中讨结煞,则自
然成就我的妙文也。

一部一百回,乃于第一回中,如一缕头发,千丝万丝,要在
头上一根绳儿扎住;又如一喷壶水,要在一提起来,即一线一
线同时喷出来。今看作者,惟西门庆一人是直说,他如应伯爵
等九人是带出,月娘三房是直叙,别的如桂姐、玳安、玉箫、子
虚、瓶儿、吴道官、天福、应宝、吴银儿、武松、武植、金莲、迎儿、
敬济、来兴、来保、王婆诸色人等,一齐皆出,如喷壶倾水。然却
是说话做事,一路有意无意,东拉西扯,便皆叙出,并非另起锅
灶,重新下米,真正龙门能事。若夫叙一人,而数人于不言中跃
跃欲动,则又神工鬼斧,非人力之所能为者矣。何以见之?如
教大丫头玉箫拿蒸酥是也。夫丫头,则丫头已耳,何以必言大
丫头哉?春梅固原在月娘房中做小丫头也,一言而春梅跃然
矣。真正化工文字。

此回内本写金莲,却先写瓶儿,妙绝。

写春梅,用影写法;写瓶儿,用遥写法;写金莲,用实写法。
然一部《金瓶》,春梅至“不垂别泪”时,总用影写,金莲总用实
写也。

写春梅,何不于首卷内直出其名哉?不知此作者特特为春
梅留身分故也。既为丫鬟,不便单单拈出,势必如玉箫借拿东
西或传话时出之,如此则春梅扫地矣。然则俟金莲进门,或云
用银自外边买来亦可。不知一部大书,全是这三个人,乃第一
回时,如何不点出也?看他于此等难处,偏能不费丝毫气力,一
笔勾出,且于不用一笔处勾出。不知其文心是天仙,是鬼怪。看
者不知,只说是拿东西赏天福,岂不大差!【2】

未出月娘,乃先插大姐,带出敬济,是何等笔力!

出敬济,止云“陈洪子”可耳,乃必云“东京八十万禁军杨
提督”者,见蔡太师、翟云峰门路,皆从此一线出来。然则又于
无笔墨处,将翟云峰、蔡太师等一齐点出矣。后文来保赂相府
时,必云“见杨府干办从府内出来”,进见蔡攸,必云“同杨干办
一齐来”,则此句出蔡京、翟云峰等益信矣。文章能事至《金瓶
梅》,真山阴道上,应接不暇,七通八达,八面玲珑,批之不尽
也。

《金瓶》内,每以一笔作千万笔用。如此回玉皇庙,谓是结
弟兄,谓是对永福寺作双峙起结,谓是出武松,谓是出金莲,谓
是笼罩“官哥寄名”、“瓶儿荐亡”等事也。总之,一笔千万用,如
神龙天际、变化不测的文字。

一回“冷”、“热”相对两截文字,然却用一笋即串拢,痕迹
俱无。所谓“笋”者,乃在玉皇庙玄坛座下一个虎,岂不奇绝!

一回两股大文字,“热结”“冷遇”也。然“热结”中七段文
字,“冷遇”中两段文字,两两相对,却在参差合笋处作对锁章
法。如正讲西门庆处,忽插入伯爵等人,至“满县都惧怕他”,下
忽接“他排行第一”,直与“复姓西门,单名一个庆字”合笋,无
一线缝处。正讲武松遇哥哥,忽插入武大别了兄弟如何如何许
多话来,下忽云“不想今日撞着自己嫡亲兄弟”,直与“自从兄
弟分别之后”合笋,无一缝处。此上下两篇文字对峙处也。

无心撞着,却是嫡亲兄弟;有心结识,反不好叙齿。掩映处
最难过,最难堪。

“热结”处,何以有七段文字?自“大宋徽宗”至“无不通
晓”是一段;自“结识的”至“都惧怕他”是两段;自“排行第一”【3】
至“又去调弄妇人”是三段;自“西门庆在家闲坐”至“只等应二
来与他说”是四段;自“正说着”至“伯爵举手和希大一路去了”
是五段;自“十月初一”至“过了初二”是六段;自“次日初三”至
“和子虚一同来家”是七段。此是“热结”的文字已毕,下文则
“冷遇”的文字了。切勿认“应伯爵来邀看虎”,犹是西门庆边的
文字。

“冷遇”两段,则一段是武大的文字,一段是金莲的文字。
伯爵两人,看去固是引子,即武松打虎见官诸事,亦是信药也。

看他写“热结”处,却用渐渐逼出。如与月娘闲话,是一顿;
伯爵、希大来相约而去,是一顿;初一日收分资,是一顿;初二
日知会道士,是一顿;初三日吃早饭,又是一顿;至庙中调笑,
又是一顿。才说吴道士请烧纸,而伯爵谦让,又作数层刷洗,方
入本题。若“冷遇”,却是一撞撞着,乃是嫡亲兄弟。便见得一
假一真,有安排不待安排处。

描写伯爵处,纯是白描追魂摄影之笔。如向希大说“何如?
我说 …… ”,又如“伸着舌头道:爷 …… ”。俨然纸上活跳出来,
如闻其声,如见其形。

《水浒》上打虎,是写武松如何踢打,虎如何剪扑;《金瓶
梅》却用伯爵口中几个“怎的”“怎的”,一个“就象是”,一个“又
象”,便使《水浒》中费如许力量方写出来者,他却一毫不费力
便了也。是何等灵滑手腕!况打虎时是何等时候,乃一拳一脚,
都能记算清白,即使武松自己,恐用力后,亦不能向人如何细
说也。岂如在伯爵口中描出为妙。

篇内出月娘,乃云“夫主面上百依百顺”。看者止知说月娘
贤德,为后文能容众妾地步也;不知作者更有深意。月娘,可以【4】
向上之人也。夫可以向上之人,使随一读书守礼之夫主,则刑
于之化,月娘便自能化俗为雅,谨守闺范,防微杜渐,举案齐
眉,便成全人矣。乃无如月娘止知依顺为道,而西门之使其依
顺者,皆非其道。月娘终日闻夫之言,是势利市井之言;见夫之
行,是奸险苟且之行,不知规谏,而乃一味依顺之,故虽有好资
质,未免习俗渐染。后文引敬济入室,放来旺进门,皆其不闻妇
道,以致不能防闲也。送人直出大门,妖尼昼夜宣卷,又其不闻
妇道,以致无所法守也。然则开卷写月娘之百依百顺,又是写
西门庆先坑了月娘也。泛泛读之,何以知作者苦心?

作者做月娘,既另出笔墨,使真欲做出一个贤妇人,后文
就不该大书特书引敬济入室等罪;既欲隐隐做他个不好的人,
又不该处处形其老实。然则写月娘,信如上所云“一个可以学
好向上的人”,西门庆不能刑于,遂致不知大礼,如俗所云“好
人到他家,也不好了”也。故“百依百顺”,是罪西门,非赞月娘。

写月娘,何以必云是继室哉?见得西门庆孤身独自,即月
娘妻子,尚是个继室,非结发者也。故其一生动作,皆是假景中
提傀儡。

写月娘恶处,又全在继室也。从来继室多是好好先生。何
则?因彼已有妻过,一旦死别,乃续一个入来,则不但他自己心
上怕丈夫疑他是填房,或有儿女,怕丈夫疑他偏心;当家,怕丈
夫疑他不如先头的。即那丈夫心中,亦未尝不有此几着疑忌在
心中。故做继室者,欲管不好,不管不好,往往多休戚不关,以
好好先生为贤也。今月娘虽说没甚奸险,然其举动处,大半不
离继室常套。故“百依百顺”,在结发则可,在继室又当别论,不
是说依顺便是贤也。是四字,又月娘定案,又继室定案。【5】

写西门对子虚,却句句是瓶儿;写子虚来入会,却又处处
是瓶儿。西门心照那边,瓶儿心照这边,已将两人十分异样亲
密处,写得花团锦簇,好看杀人。真有笔不到而意到之妙。

凡人用笔曲处,一曲两曲足矣,乃未有如《金瓶》之曲也。
何则?如本意欲出金莲,却不肯如寻常小说云“按下此处不言,
再表一个人,姓甚名谁”的恶套。乃何如下笔?因思从兄弟“冷
遇”处带出金莲;然则如何出此两兄弟?则用先出武二;如何出
武二?则用打虎;如何出打虎?是依旧要先出武二矣。否则依
旧要按下此处,再讲清河县出示拿虎矣。夫费如许曲折,乃依
旧要按下另讲,文章之夯,亦夯不至此。不知作者乃眼觑一处
矣。何则?玉皇庙固黄河发源之所,瓶儿既于此处出,金莲能
不于此处出哉!故一眼觑见玉皇庙四大元帅,作者不觉搁笔拍
案大笑也。然而其下笔时,偏不即写玄坛,乃先写老子青牛,又
写二重殿,又写侧门,又写正面三间厂厅,又写昊天上帝,又写
紫府星官,方出四大元帅。文至此,所谓曲折亦曲折尽矣。看
他偏不即写玄坛,乃又先写马元帅,带出帮闲讨好,使本文“热
结”中意思柳遮花映,八面玲珑。至此该写赵元帅矣,偏又不肯
写下,又放过赵元帅,再写温元帅,又照入帮闲身分,放倒自
己,奉承他人。使“热结”本文不脱生,十分美满后,才又插转玄
坛,玄坛身边,方出画虎。曲折至此,该用吴道官说出真虎矣,
乃偏又漾开,偏又照管众帮闲,点染“热结”本文,方用吴道官
一点真虎。夫所谓打虎之人,尚查然不知音信。止因一个画虎,
便如此曲折,真不怕呕血,不怕鬼哭。文至此,可云至矣。看他
偏有力量,偏又照入打虎情景;在白赉光口中,偏又令伯爵又
插一笑谈,花遮柳映,又照入“热结”本文来。夫写一面照一面,【6】
犹他人所能,乃于写这一面时,却是写那一面,写那一面时,却
原是写这一面。七穿八达,出神入化,所谓不怕呕血,不怕鬼
哭,是真不怕呕血、鬼哭者矣。盖人一手写一处不能,他却一手
写三四处也。玉皇庙是一处,十弟兄是一处,道士是一处,画虎
是一处,真虎是一处,打虎人又遥在一处,跃然欲动,而沧州郡
且明明说出也。后生家看此等文字,而不心灰气绝,回家焚烧
笔砚,再不敢做文者,是必目不识丁,卖菜佣不如之人也。

夫不有子虚,则瓶儿归西门,是无孽之人矣,故必有子虚;
然子虚不虽有如无,则瓶儿又何以归西门?是故子虚是个影子
中人。今于影子中人上场,不加一番描写渲染,则何以见其为
影子中人哉?故曰于排次第时见之矣。何则?若论势字当从
财主,西门庆家不是世代阀阅,止因有几贯钱,方能使势也。夫
既以钱为主,子虚之钱较西门为加倍,如此应该子虚为大,乃
不但不能僭西门之左,且不能居应、谢二人之上;而应、谢二
人,明明知其财主,亦绝不相让,则子虚为虽有如无之人不言
已喻,而财必至为他人之财,妻必至为他人之妻,此时已定局
矣。故无论他盈千累万的家财,必先看他有好儿子没有,才定
得是他的不是他的。文字妙处,全要在不言处见。试问看官:
有几个看没字处的人否?

一回内句句“三娘”,而玉楼亦跃跃纸上,此所开缺候官之
法也。

写虎一段,自入三间厂厅内,一引入,一漾开,凡三四折,
方入吴道官。文字又如穿花蝴蝶,一远一近,煞是好看杀人。

“热结”文字,却以花二娘起,花二娘结,而月娘作引,卓二
姐作余波。人只谓下文是瓶儿先讲起,不知一渡即是金莲文【7】
字。作者之笔,其如龙乎!看他每不肯为人先算着。

西门庆“沉吟一会”,乃说出花子虚来。试想其沉吟是何意
思?直与九回中武二沉吟一会相照。西门一沉吟,子虚死矣。
武二一沉吟,李外传、王婆、金莲俱死矣,而西门庆亦死矣。然
武二沉吟是杀人,西门沉吟是自杀。

写金莲,云“不知这妇人是个使女出身”,后文瓶儿出身,
又是“梁中书侍妾”,春梅不必说矣。然则三人大抵皆同。作者
盖深恶此等人,亦见婢妾中邪淫者多也。

“冷遇”哥嫂文中,乃一云“嫡亲兄弟”,再云“是我一母同
胞兄弟”,再云“亲兄弟难比别人”。句句是武二文字,却句句是
敲击十兄弟文字也。

篇内金莲凡十二声“叔叔”,于十一声下,作者却自入一
句,将上文十一声“叔叔”一总,下又拖一句“叔叔”,便见金莲
心头眼底口中,一时便有无数“叔叔”也。益悟文章生动处,不
在用笔写到之处。

开卷一部大书,乃用一律、一绝、三成语、一谚语尽之,而
、又入四句偈作证,则可云《金瓶梅》已告完矣。

《水浒》本意在武松,故写金莲是宾,写武松是主。《金瓶
梅》本写金莲,故写金莲是主,写武松是宾。文章有宾主之法,
故立言体自不同,切莫一例看去。所以打虎一节,亦只得在伯
爵口中说出。

“里仁为美”,况近邻哉!今子虚不善择邻,而与西门为邻,
卒受其祸;武大与王婆为邻,亦卒受其祸;殆后瓶儿与金莲邻.
墙,又卒受其祸。甚矣,卜邻当慎也!【8】

\banben{一解}[崇本={(无)}]:(张行评:上解空去财。)

豪华去后行人绝,箫筝不响歌喉咽。

雄剑无威光彩沉,宝琴零落金星灭。

\banben{二解}[崇本={(无)}]:(张行评:下解空去色。)

玉阶寂寞坠秋露,月照当时歌舞处。

当时歌舞人不回,化为今日西陵灰。(崇眉评:一部炎凉景
况,尽在此数语中。)

\banben{色箴}[崇本={(无)}]:

二八佳人体似酥,腰间仗剑斩愚夫;

虽然不见人头落,暗里教君骨髓枯。

这一首诗,是昔年大唐国时,一个修真炼性的英雄,入圣超凡
的豪杰,到后来位居紫府,名列仙班,率领上八洞群仙,救拔四部洲
沉苦一位仙长,姓吕名岩,道号纯阳子祖师所作。单道世上人营营
逐逐、急急巴巴,跳不出七情六欲关头,打不破酒色财气圈子,到头
来同归于尽,着甚要紧。(张行评:以上总起四字,借一吕纯阳作开
讲,奇绝。所以有后文吴神仙、黄真人、潘道士也。)

虽是如此说,只这酒色财气四件中,惟有“财色”二者更为利
害。怎见得他的利害?假如一个人到了那穷苦的田地,受尽无限凄
凉,耐尽无端懊恼,晚来摸一摸米瓮,苦无隔宿之炊,早起看一看厨
前,愧没半星烟火,(崇眉评:情景逼真。酸侠读此,能不雪涕!)妻子
饥寒,一身冻馁,就是那粥饭尚且艰难,那讨余钱沽酒?(崇旁评:酒
因财缺。)更有一种可恨处,亲朋白眼,面目寒酸,便是凌云志气,分
外消磨,怎能勾与人争气!(崇旁评:气以财弱。 张行评:以上反
起财。)正是:(张行评:这一个“正是”是冷。)

一朝马死黄金尽,亲者如同陌路人。(张行评:财箴。)【9】

\noindent 到得那有钱时节,挥金买笑,一掷巨万。思饮酒,真个琼浆玉液,(崇
旁评:酒需财美。)不数那琥珀杯流;要斗气,钱可通神,果然是颐指
气使。(崇旁评:气用财伸。)趋炎的压脊挨肩,附势的吮痈舐痔,(张
行评:以上正说财。)真所谓得势叠肩来,失势掉臂去。古今炎凉恶
态,莫有甚于此者。这两等人,岂不是受那财的利害处!(张行评:
此下共作四扇股法,色一股,财一股,看破的财一股,看破的色一
股。而上二股内,乃各插入酒、气二种,盖本意只重财、色,而又借
酒、气串入。股法生动不板也。)

如今再说那色的利害。请看如今世界,你说那坐怀不乱的柳下
惠,闭门不纳的鲁男子,与那秉烛达旦的关云长,古今能有几人?
(崇眉评:引起三段,格法一变,更见灵活。张行评:三个不怕色的
人做好样。)至如三妻四妾,买笑追欢的,又当别论。还有那一种好
色的人,见了个妇女略有几分颜色,便百计千方偷寒送暖,一到了
着手时节,只图那一瞬欢娱,也全不顾亲戚的名分,也不想朋友的
交情。起初时,不知用了多少滥钱,费了几遭酒食。(崇旁评:酒。)
正是:(张行评:这一个“正是”是热。)

三杯茶作合,两盏色媒人。(张行评:酒箴。)

到后来情浓事露,甚有斗狠杀伤,(崇旁评:气。)性命不保,妻孥难
顾,事业成灰。就如那石季伦泼天豪富,为绿珠命丧囹圄;楚霸王气
概拔山,因虞姬头悬垓下。(张行评:两个不胜色的人做歹样。)真所
谓“生我之门死我户,看得破时忍不过”。这样人,岂不是受那色的
利害处?(张行评:两“岂不是”,章法奇绝对峙。)

说便如此说,这“财色”二字,从来只没有看得破的。若有那看
得破的,(张行评:又单一句另起。)便见得堆金积玉,是棺材内带不
去的瓦砾泥沙;贯朽粟红,是皮囊内装不尽的臭污粪土;高堂广厦,【10】
玉宇琼楼,是坟山上起不得的享堂;\banben{锦衣绣袄}[崇本={锦衣绣裙}],狐服貂裘,是骷髅上
裹不了的败絮。(崇眉评:说得世情冰冷,须坐蒲团面壁十年才辨。
 张行评:看破后的财,七十九回已后之财也。)即如那妖姬艳女,
献媚开妍,看得破的,却如交锋阵上将军叱咤献威风;朱唇皓齿,掩
袖回眸,懂得来时,便是阎罗殿前鬼判夜叉增恶态。罗袜一湾,金莲
三寸,是砌坟时破土的锹锄;(崇旁评:尖颖异常。)枕上绸缪,被中
恩爱,是五殿下油锅中生活。(张行评:看破后的色,七十九回已后
之色也。)只有那《金刚经》上两句说得好,他说道:“如梦幻泡影,如
电复如露。”(张行评:是一部大主意,大结果,大解脱,所以有普净
也。)见得人生在世,一件也少不得,到了那结果时,一件也用不着。
(张行评:又单一句,与上“看破”句作对。)随着你举鼎荡舟的神力,
到头来少不得骨软筋麻;(张行评:虚陪一句。)由着你铜山金谷的
奢华,正好时却又要冰消雪散。(张行评:为西门庆说法。)假饶你闭
月羞花的容貌,一到了垂眉落眼,人皆掩鼻而过之;(张行评:为金
莲辈说法。)比如你\banben{(淫)}[崇本={〔陆〕}]贾、隋何的机锋,若遇着齿冷唇寒,吾末如之
何也已。(崇眉评:生公说法,石应首肯。张行评:为伯爵辈说法。)
到不如削去六根清净,(崇旁评:伏脉。)披上一领袈裟,参透了空色
世界,打磨穿生灭机关,直超无上乘,不落是非窠,倒得个清闲自
在,不向火坑中翻筋斗也。(张行评:为普净作案。)正是:(张行评:
这一个“正是”,是冷热俱无。)

三寸气在千般用,一旦无常万事休。(张行评:气箴。)

说话的,为何说此一段酒色财气的缘故?只为当时有一个人
家,先前恁地富贵,到后来煞甚凄凉,权谋术智,一毫也用不着,亲
友兄弟,一个也靠不着,享不过几年的荣华,倒做了许多的话靶。内
中又有几个斗宠争强,迎奸卖俏的,起先好不妖娆妩媚,到后来也【11】
免不得尸横灯影,血染空房。(张行评:此一段是一部小《金瓶》,如
世所云总纲也。)正是:(张行评:这一个“正是”,是天不肯使人冷热
到地。)

善有善报,恶有恶报;

天网恢恢,疏而不漏。

(张行评:以上一部大书总纲,此四句又总纲之总纲。信乎
《金瓶》之纯体天道以立言也。)

话说大宋徽宗皇帝政和年间,(张旁评:记清。)山东省东平府
清河县中,(张旁评:记清。)有一个风流子弟,生得状貌魁梧,(张行
评:病根一。)性情潇洒,(张行评:病根二。)饶有几贯家资,(张行
评:病根三。)年纪二十六七。这人覆姓西门,单讳一个庆字。他父
亲西门达,原走川广贩卖药材,就在这清河县前开着一个大大的生
药铺。现住着门面五间到底七进的房子,(张旁评:记清。)家中呼奴
使婢,骡马成群,虽算不得十分富贵,(张行评:为后得几主横财生
子加官地步。)却也是清河县中一个殷实的人家。(张行评:为后奢
华反照。)只为这西门达员外夫妇去世的早,单生这个儿子,却又百
般爱惜,听其所为。(张行评:是不读书的病根。)所以这人不甚读
书,(崇旁评:四字是一生病痛。 张行评:大书特书一部作孽的病
根。)终日闲游浪荡。一自父母亡后,专一在外眠花宿柳,惹草招风,
学得些好拳棒,又会赌博,双陆、象棋、抹牌、道字,无不通晓。(张行
评:是他一付作业的本事,预先说明。)结识的朋友,也都是些帮闲
抹嘴,不守本分的人。第一个最相契的,姓应名伯爵,表字光侯,(张
行评:应伯爵如此出法,所谓抹嘴者也。)原是开绸段铺应员外的第
二个儿子,落了本钱,跌落下来,专在本司三院帮嫖贴食,因此人都
起他一个诨名,叫做应花子;又会一腿好气球,双陆、棋子件件皆【12】
通。(崇眉评:叙得错综变化。)第二个姓谢,名希大,字子纯,(张行
评:谢希大如此出法,所谓帮闲者也。)乃清河卫千户官儿应袭子
孙,自幼父母双亡,游手好闲,把前程丢了,亦是帮闲勤儿,会一手
好琵琵。自这两个与西门庆甚合得来。(张行评:一束二人,再叙下
八人,文字错落有致。)其余还有几个,都是些破落户,没名器的。一
个叫做祝实念,表字贡诚。一个叫做孙天化,表字伯修,绰号孙寡
嘴。一个叫做吴典恩,乃是本县阴阳生,因事革退,专一在县前与官
吏保债,以此与西门庆往来。(张行评:顺手为放债一照。)还有一个
云参将的兄弟,叫做云理守,字非去。一个叫做常峙节,表字坚初。
一个叫卜志道。一个叫做白赉光,表字光汤。说这白赉光,众人中
也有道他名字取的不好听的,他却自己解说道:“不然我也改了,只
为当初取名的时节,原是一个门馆先生说我姓白,当初有一个甚么
故事,是白鱼跃入武王舟;又说有两句书是‘周有大赉,于汤有光’,
取这个意思,所以表字就叫做光汤。我因他有这段故事,也便不改
了。”(崇眉评:磊落写来,于结处独以此段潆洄,便觉须眉生动。
张行评:看他叙出十弟兄,虽一篇小小文章,却参差错落,而与西门
庆亲疏厚薄,以及后文各人的行事、终身,皆不烦言而毕见,真化工
之笔也,惟古史迁可以似之。)说这一干共十数人,见西门庆手里有
钱,又撒漫肯使,所以都乱撮哄着他耍钱饮酒,嫖赌齐行。正是:
把盏衔杯意气深,兄兄弟弟抑何亲。

一朝平地风波起,此际相交才见心。(张行评:总起西门交
游。)

说话的,这等一个人家,生出这等一个不肖的儿子,又搭了这
等一班无益有损的朋友,随你怎的豪富也要穷了,还有甚长进的日
子?却又有一个缘故,只为这西门庆生来秉性刚强,作事机深诡谲,【13】
又放官吏债,就是那朝中高、杨、童、蔡四大奸臣,他也有门路与他
浸润,(崇眉评:好针线。)所以专在县里管些公事,与人把揽,说事
过钱,因此满县人都惧怕他。因他排行第一,人都叫他是西门大官
人。这西门大官人先头浑家陈氏早逝,身边止生得一个女儿叫做西
门大姐,就许与东京八十万禁军杨提督的亲家陈洪的儿子陈敬济
为室,(张行评:说西门浸润下,接手叙出大姐、敬济。盖明陈洪者,
西门浸润之门路也。因陈洪而通杨戬,因杨戬而通蔡京,故大姐、敬
济后报独惨。)尚未过门;只为亡了浑家,无人管理家务,新近又娶
了本县清河左卫吴千户之女,填房为继室。这吴氏年纪二十五六,
是八月十五生的,(张行评:注明秋月。)小名叫做月姐,后来嫁到西
门庆家,都顺口叫他月娘。\banben{却说这月娘}[崇本={(无)}]秉性贤能,夫主面上百依百
随。(崇眉评:如此贤妇,世上有几? 张行评:二语全为西门罪,不
是赞月娘也,已于卷首讲明。)房中也有三四个丫鬟妇女,都是西门
庆收用过的。(张行评:伏雪娥、玉箫诸人。)又尝与勾栏内李娇儿打
热,也娶在家里,做了第二房娘子。南街又占着窠子卓二姐,名卓丢
儿,包了些时,也娶来家做了第三房。只为卓二姐身子瘦怯,时常三
病四痛,(张行评:以上正是三房妻妾,却是两实一虚。)\banben{却又去}[崇本={他却又去}]飘风
戏月,调弄人家妇女。(张行评:文气至此一顿,叙完西门出身,是一
篇小文字。)

东家歌笑醉红颜,又向西邻开玳筵。

几日碧桃花下卧,牡丹开处总堪怜。(张行评:总起西门淫
孽。)

话说西门庆一日在家闲坐,对吴月娘说道:“如今是九月廿五
日了,(张行评:九月廿五日起头,九月十七日瓶儿死,自七至五,中
余七日,七日来复之义。西门三十三岁,正月廿一日死。三十三老【14】
阳,廿一少阳。老变少,所以有孝哥也。)出月初三日,却是我兄弟们
的会期,到那日,也少不的要整两席齐整酒席,叫两个唱的姐儿,自
恁在咱家与兄弟们好生顽耍一日。你与我料理料理。”吴月娘便道:
“你也便别要说起这干人,那一个是那有良心的行货?无过每日来
勾使的游魂撞尸。我看你自搭了这起人,几时曾着个家哩?(崇眉
评:数语可配名臣谏疏。 张行评:逆入热结。)现今卓二姐自恁不
好,我劝你把那酒也少要吃了。”西门庆道:“你别的话倒也中听,今
日这些说话,我却有些不耐烦听他。依你说,这些兄弟们没有好人,
别的倒也罢了,自我这应二哥这一个人,本心又好又知趣,(崇旁
评:溺爱者智昏,不止西门一个。)着人使着他,没有一个不依顺的,
做事又十分停当。(张行评:将后文荐引诸伙计与说诸事,俱提出。
内有王六儿诸人在也。)就是那谢子纯这个人,也不失为个伶俐能
事的好人。(张行评:又陪希大一句。)咱如今是这等计较罢,只管恁
会来会去,终不着个切实。咱不如到了会期,都结拜了兄弟罢,明日
也有个靠傍些。”吴月娘接过来道:“结拜兄弟也好,只怕后日还是
别个靠的你多哩!若要你去靠人,提傀儡儿上戏场——还少一口气
儿哩。”西门庆笑道:“咱恁长把人靠得着,却不更好了。咱只等应二
哥来,与他说这话罢。”(张行评:出结拜,又是这等出法。)

正说着话,只见一个小厮儿,生得眉清目秀,伶俐乖觉,原是西
门庆贴身伏侍的,唤名玳安儿,走到面前来说:“应二叔和谢大叔在
外,见爹说话哩。”(张行评:顺手出玳安。)西门庆道:“我正说他,他
却两个就来了。”一面走到厅上来。只见应伯爵头上戴一顶新盔的
玄罗帽儿,身上穿一件半新不旧的天青夹绉纱褶子,脚下丝鞋净
袜,坐在上首。下首坐的便是姓谢的谢希大。(张行评:希大处处陪
写,故名希大。)见西门庆出来,一齐立起身来,连忙作揖道:“哥在【15】
家,连日少看。”西门庆让他坐下,一面唤茶来吃,说道:“你们好人
儿!这几日我心里不耐烦,不出来走跳,你们通不来傍个影儿。”(张
行评:试问出笔不如此,却如何开口?)伯爵向希大道:“何如?我说
哥要说哩!”(张行评:妙。纯是白描,却是放重笔拿轻笔法,切须学
之也。)因对西门庆道:“哥,你怪的是。连咱自也不知道成日忙些什
么。自咱们这两只脚,还赶不上一张嘴哩!”西门庆因问道:“你这两
日在那里来?”伯爵道:“昨日在院中李家,瞧了个孩子儿,就是哥这
边二嫂子的侄女儿,(张旁评:一重亲。)桂卿的妹子,(张旁评:一重
亲。)叫做桂姐儿。几时儿不见他,就出落的好不标致了!到明日成
人的时候,还不知怎的样好哩。昨日他妈再三向我说:‘二爹千万寻
个好子弟梳笼他。’敢怕明日还是哥的货儿哩。”(崇旁评:伏脉。 
张行评:带出桂姐。)西门庆道:“有这等事?等咱空闲了去瞧瞧。”谢
希大接过来道:“哥不信,委的生得十分颜色。”(张行评:希大说话,
通是随着伯爵,至篇终皆如此,故不犯伯爵也。)西门庆道:“昨日便
在他家,前几日却在那里去来?”伯爵道:“便是前日,卜志道兄弟死
了;咱在他家帮着乱了几日,发送他出门。(崇旁评:伏脉。)他娘子
再三向我说,叫我拜上哥,承哥这里送了香楮奠礼去,因他没有宽
转地方儿,晚夕又没甚好酒席,不好请哥坐的,甚是过不意去。”西
门庆道:“便是我闻得他不好得没多日子,就这等死了。我前日承他
送我一把真金川扇儿,我正要拿甚答谢答谢,不想他又做了故人!”
(张行评:既云兄弟,乃于生死时只如此,冷淡杀人。写十兄弟身分,
如此一笔,直照西门死后也。)

谢希大便叹了一口气道:“咱会中兄弟十人,却又少他一个
了。”因向伯爵说:“出月初三日,又是会期,咱每少不得又要烦大官
人这里破费,兄弟们顽耍一日哩。”(张行评:希大说出,便不及伯爵【16】
一步,所以妙也。)西门庆便道:“正是。我刚才正对房下说来,咱兄
弟们似这等会来会去,无过只是吃酒顽耍,不着一个切实,倒不如
寻一个寺院里,写上一个疏头,结拜做了兄弟,到后日彼此扶持,有
个靠傍。到那日,咱少不得要破些银子,买办三牲,众兄弟也便随多
少各出些分资。不是我科派你们,这结拜的事,各人出些,也见些情
分。”(张行评:是大老官口吻。)伯爵连忙道:“哥说的是。婆儿烧香,
当不的老子念佛,各自要尽自的心。(张行评:一承认。)只是俺众人
们,‘老鼠尾靶生疮儿——有脓也不多’。”(张行评:便自谦,写尽帮
闲丑态。)西门庆笑道:“怪狗才,谁要你多来?你说这话!”谢希大
道:“结拜,须得十个方好。(张行评:必须十个妙。如此方是这班人
结拜也。)如今卜志道兄弟没了,却教谁补?”西门庆沉吟了一回,说
道:(张行评:试想其沉吟为何?一个花二娘已在其沉吟中也。妙
绝。)“咱这间壁花二哥,原是花太监侄儿,手里肯使一股滥钱,(张
行评:伏后转元宝。)常在院中走动,他家后边院子,与咱家只隔着
一层壁儿,(崇旁评:伏脉。)与我甚说得来,咱不如叫小厮邀他邀
去。”(张行评:算出子虚。)应伯爵拍着手道:“敢就是在院中包着吴
银儿的花子虚么?”(张行评:顺出银儿。)西门庆道:“正是他。”伯爵
笑道:“哥快叫那个大官儿邀他去,(崇旁评:等不得了。)与他往来
了,咱到日后敢又有一个酒铺儿。”西门庆笑道:“傻花子,你敢害馋
痨痞哩,说着的是吃。”大家笑了一回。

西门庆旋叫过玳安儿来说:“你到间壁花家去,对你花二爹说,
如此这般:‘俺爹到出月初三日要结拜十兄弟,敢叫我请二爹上会
哩。’看他怎的说,你就来回我话。你二爹若不在家,就对他二娘说
罢。”(张行评:巧出瓶儿,此沉吟之故也,所以必拉他上会。)玳安儿
应诺去了。伯爵便道:“到那日,还在哥这里,是还在寺院里好?”希【17】
大道:“咱这里无过只两个寺院,僧家便是永福寺,道家便是玉皇
庙。(崇旁评:又伏永福寺、玉皇庙。)这两个去处,(张行评:玉皇庙、
永福寺,须记清白,是一部起结也,明明说出全以二处作终始的柱
子,乃俗批伏出。可笑,可笑。)随分那里去罢。”西门庆道:“这结拜
的事,不是僧家管的,(张行评:然则道家单管结拜乎?写愚处,映不
读书。)那寺里和尚我又不熟,倒不如玉皇庙吴道官与我相熟,他那
里又宽厂,又幽静。”伯爵接过来道:“哥说的是。敢是永福寺和尚倒
和谢家嫂子相好?故要荐与他去的。”(张行评:虽随手成趣,亦映带
讲花二娘心事。)希大笑骂道:“老花子!一件正事,说说就放出屁来
了。”

正说笑间,只见玳安儿转来了,因对西门庆说道:“他二爹不在
家,(张行评:此作者为要出瓶儿也,若说真个不在家,岂不大呆。)
俺对他二娘说来。二娘听了,好不欢喜,(崇旁评:伏脉。)说道:‘既
是你西门爹携带你二爹做兄弟,那有个不来的!(崇眉评:只恐携带
二爹,便要插戴二娘。)等来家我与他说。(张行评:又写瓶儿作得
主,以照下文。)至期一定撺掇他来,多拜上爹。’(张行评:四字妙
绝,正对“沉吟”。)又与了小的两件茶食来了。”(崇旁评:闲处都韵。
 张行评:又写瓶儿为人处,照下。)西门庆对应、谢二人道:“自这
花二哥,倒好个伶俐标致娘子。”(崇旁评:伏脉。)方说毕,又拿一盏
茶吃了,二人一齐起身道:“哥,别了罢,咱自去通知众兄弟,纠他分
资来。哥这里先去与吴道官说声。”西门庆道:“我知道了,我也不留
你罢。”于是一齐送出大门来。应伯爵走了几步,回转来道:“那日可
要叫唱的?”西门庆道:“这也罢了,弟兄们说说笑笑,到有趣些。”说
毕,伯爵举手和希大一路去了。(张行评:须知此段文字,全为子
虚。)【18】

话休饶舌。捻指过了四五日,却是十月初一日。(张行评:初一
日又起。)西门庆早起,刚在月娘房里坐的,只见一个才留头的小厮
儿,(张行评:天福也者。)手里拿着个描金退光拜匣,(张眉评:一拜
匣而子虚殷实如见。)走将进来,向西门庆磕了一个头儿,立起来,
站在傍边,说道:“俺是花家,俺爹多拜上西门爹。那日西门爹这边
叫大官儿请俺爹去,俺爹有事出门了,不曾当面领教的。闻得爹这
边是初三日上会,俺爹特使小的先送这些分资来,说爹这边胡乱先
用着,等明日爹这里用过多少,派开,该俺爹多少,再补过来便了。”
西门庆拿起封袋一看,签上写着“分资一两”,便道:“多了,不消补
的。到后日叫爹莫往那去,起早就要同众爹上庙去。”那小厮儿应
道:“小的知道。”刚待转身,被吴月娘唤住,(崇旁评:临去秋波。)叫
大丫头玉箫在食箩里拣了两件蒸酥果馅儿与他。(张行评:又出玉
箫,为春梅一影,不然何以云大丫头也?影出春梅。)因说道:“这是
与你当茶的。你到家拜上你家娘,你说西门大娘说,迟几日还要请
娘过去坐半日儿哩。”(崇旁评:想必要结十姊妹。)那小厮接了,又
磕了个头儿,应着去了。

西门庆才打发花家小厮出门,只见应伯爵家应宝夹着个拜匣,
玳安儿引他进来见了,磕了头,说道:“俺爹纠了众爹们分资,叫小
的送来,爹请收了。”西门庆取出来看,共总八封,也不拆看,都交与
月娘道:“你收了,到明日上庙,好凑着买东西。”说毕,打发应宝去
了。立起身到那边看卓二姐。刚走到坐下,只见玉箫走来,说道:
“娘请爹说话哩。”(崇旁评:余波。)西门庆道:“怎的起先不说来?”
随即又到上房,看见月娘摊着些纸包在面前,指着笑道:“你看这些
分子,止有应二的是一钱二分,八成银子,其余也有三分的,也有五
分的,都是些红的、黄的,倒像金子一般。咱家也曾没见这银子来?【19】
收他的也污个名,不如掠还他罢。”(张行评:又应出十兄弟身分。追
魂摄影之笔也。)西门庆道:“你也耐烦,丢着罢,咱多的也包补,在
乎这些?”说着一直往前去了。(张行评:又一顿。)

到了次日初二日,(张行评:初二日。)西门庆称出四两银子,叫
家人来兴儿(张行评:来兴儿。)买了一口猪、一口羊、五六坛金华酒
和香烛纸扎、鸡鸭案酒之物,又封了五钱银子,旋叫了大家人来保
(张行评:来保儿必云大家人,后文俱出。)和玳安儿、来兴三个:“送
到玉皇庙去,对你吴师父说,俺爹明日结拜兄弟,要劳师父做纸疏
辞,晚夕就在师父这里散福。烦师父与俺爹预备预备,俺爹明早便
来。”只见玳安儿去了一会,来回说:“已送去了,吴师父说知道了。”

须臾,过了初二。(张行评:又一顿。)次日初三早,(张行评:初
三。)西门庆起来,梳洗毕,叫玳安儿:“你去请花二爹,到咱这里吃
早饭,一同好上庙去。(张行评:心在瓶儿。)一发到应二叔家,叫他
催催众人。”玳安应诺去。刚请花子虚到来,只见应伯爵和一班兄弟
也来了,却正是前头所说的这几个人。(崇旁评:照出。)为头的便是
应伯爵,谢希大、孙天化、祝实念、吴典恩、云理守、常峙节、白赉光,
连西门庆、花子虚共成十个。进门来一齐箩圈作了一个揖。伯爵道:
“这时候好去了。”西门庆道:“也等吃了早饭着。”便叫:“拿茶来。”
一面叫:“看菜儿。”须臾,吃毕早饭,(张行评:又一顿,文字细极。)
西门庆换了一身衣服,打选衣帽光鲜,一齐径往玉皇庙来。

不到数里之遥,早望见那座庙门,造得甚是雄峻。但见:

殿宇嵯峨,宫墙高耸。正面前,起着一座墙门八字,一带都
粉赭色红泥;进里边,列着三条甬道川纹,四方都砌水痕白石。
正殿上金碧辉煌,两廊下檐阿峻峭。三清圣祖庄严宝相列中
央,太上老君背倚青牛居后殿。【20】

进入第二重殿后,转过一重侧门,却是吴道官的道院。进的门来,两
下都是些瑶草琪花,苍松翠竹。西门庆抬头一看,只见两边门楹上
贴着一副对联道:

洞府无穷岁月

壶天别有乾坤

\noindent 上面三间厂厅,却是吴道官朝夕做作功课的所在。当日铺设甚是齐
整,上面挂的是昊天金阙玉皇上帝,(张旁评:一个陪客。)两边\banben{挂着
的紫府星官}[崇本={列着的紫府星官}],(张旁评:两个陪客。)侧首挂着便是马、赵、温、黄四大
元帅。(崇旁评:伏脉。 张旁评:引入。)

当下吴道官却又在经堂外躬身迎接。西门庆一起人进入里边,
献茶已罢,众人都起身,四围观看。白赉光携着常峙节手儿,从左边
看将过来,(张旁评:有层次。)一到马元帅面前,见这元帅威风凛
凛,相貌堂堂,面上画着三只眼睛,便叫常峙节道:“哥,这却是怎的
说?如今世界,开只眼闭只眼儿便好,还经得多出只眼睛看人破绽
哩!”应伯爵听见,走过来道:“呆兄弟,他多只眼儿看你倒不好么?”
(崇旁评:隽。张行评:先点西门。)众人笑了。常峙节便指着下首
温元帅道:“二哥,这个通身蓝的,却也古怪,敢怕是卢杞的祖宗?”
伯爵笑着猛叫道:“吴先生,你过来,我与你说个笑话儿。”那吴道官
真个走过来听他。伯爵道:“一个道家,死去见了阎王。阎王问道:
‘你是甚么人?’道者说:‘是道士。’阎王叫判官查他,果系道士,且
无罪孽。‘这等,放他还魂。’只见道士转来,路上遇着一个染坊中的
博士,原认得的,那博士问道:‘师父,怎生得转来?’道者说:‘我是
道士,所以放我转来。’那博士记了,见阎王时也说是道士。那阎王
叫查他身上,只见伸出两只手来,是蓝的。问其何故?那博士打着
宣科的声音道:‘曾与温元帅搔胞。’”(张行评:伯爵辈写照。)说的【21】
众人大笑。一面又转过右首来,见下首\banben{供着个黄将军,威风凛凛}[词本={供着个红脸的,却是关帝}]。上
首又是一个黑面的,是赵元坛元帅,身边画着一个大老虎。(张旁
评:又引入。)白赉光指着道:“哥,你看这老虎,难道是吃素的,随着
人不妨事么?”伯爵笑道:“你不知,这老虎是他一个亲随的伴当儿
哩。”谢希大听得,走过来,伸着舌头道:“这等一个伴当随着我,一
刻也成不的。我不怕他要吃我么!”伯爵笑着向西门庆道:“这等,亏
他怎地过来!”西门庆道:“却怎的说?”伯爵道:“子纯一个要吃他的
伴当随不的,似我们这等七八个要吃你的随你,却不吓死了你罢
了。”(崇旁评:趣。张行评:总写十兄弟。)说着,一齐正大笑时,吴
道官走过来,(崇眉评:落脉无痕,手笔入化。)说道:“官人们讲这老
虎,只俺这清河县,这两日好不受这老虎的亏!往来的人也不知吃
了多少,就是猎户,也害死了十来人。”西门庆问道:“是怎的来?”吴
道官道:“官人们还不知道?不然我也不晓的,只因日前一个小徒,
到沧州横海郡柴大官人那里,去化些钱粮,(崇旁评:照应。)整整住
了五七日,才得过来。俺这清河县近着沧州路上,有一条景阳岗,岗
上新近出了一个吊睛白额老虎,时常出来吃人。客商过往,好生难
走,必须要成群结伙而过。如今县里现出着五十两赏钱,要拿他,白
拿不得。可怜这些猎户,不知吃了多少限棒哩!”白赉光跳起来道:
“咱今日结拜了,明日就去拿他,也得些银子使。”西门庆道:“你性
命不值钱么?”白赉光笑道:“有了银子,要性命怎的!”众人齐笑起
来。应伯爵道:“我再说个笑话你们听:(张旁评:又荡开。)一个人被
虎衔了,他儿子要救他,拿刀去杀那虎。这人在虎口里叫道:‘儿子,
你\banben{省可儿}[崇本={省可尔}]的砍。怕砍坏了虎皮。’”(崇眉评:这才是要钱不要命。)说
着,众人哈哈大笑。(张行评:自“上面三间”至此,总是为“冷遇”作
楔子,不是“热结”中文字。)【22】

只见吴道官打点牲礼停当,来说道:“官人们烧纸罢。”一面取
出疏纸来,说:“疏已写了,只是那位居长?那位居次?排列了,好等
小道书写尊讳。”(张行评:至此才叙“热结”正文。)众人一齐道:“这
自然是西门大官人居长。”(崇旁评:怎见得?张旁评:目中全无子
虚。)西门庆道:“这还是叙齿,应二哥大如我,是应二哥居长。”伯爵
伸着舌头道:“爷,可不折杀小人罢了!(崇眉评:小人,一副行乐
图。)如今年时,只好叙些财势,那里好叙齿?(崇旁评:可怜,可叹!)
若叙齿,还有大如我的哩。且是我做大哥,有两件不妥:第一,不如
大官人有威有德,(崇旁评:要紧话。)众兄弟都服你;第二,我原叫
应二哥,如今居长,却又要叫应大哥了,(张行评:言下已反衬子虚
没用,故伯爵自己先认第二坐矣。)倘或有两个人来,一个叫‘应二
哥’,一个叫‘应大哥’,我还是应‘应二哥’应‘应大哥’呢?”(崇行
评:小人媚势,亦自有说。)西门庆笑道:“你这挡断肠子的,单有这
些闲说的!”谢希大道:“哥休推了。”西门庆再三谦让,被花子虚、应
伯爵等一干人逼勒不过,只得做了大哥。第二便是应伯爵,第三谢
希大,第四让花子虚,有钱做了四哥。(张行评:有钱且居第四,总写
子虚不堪。)其余挨次排列。吴道官写完疏纸,于是点起香烛,众人
依次排列。吴道官伸开疏纸,朗声读道:

维大宋国山东东平府清河县信士(张行评:妙,然则不过
作成吴道官一次耳。)西门庆、应伯爵、谢希大、花子虚、孙天
化、祝实念、云理守、吴典恩、常峙节、白赉光等,是日沐手焚
香,请旨。伏为桃园义重,众心仰慕而敢效其风;管鲍情深,各
姓追维而欲同其志。况\banben{四海皆可兄弟}[崇本={四海皆可弟兄}],岂异姓不如骨肉?是以
\banben{当今}[崇本={涓今}]政和年月日,虔备猪羊牲礼,鸾驭金资,耑叩\banben{(齐)}[崇本={〔斋〕}]斋坛,虔诚请
祷,拜投昊天金阙玉皇上帝,五方直日功曹,本县城隍社令,过【23】
往一切神衹,仗此真香,普同鉴察。伏念庆等,生虽异日,死冀
同时。期盟言之永固,安乐与共,颠沛相扶。思缔结以常新,必
富贵常念贫穷,乃始终有所依倚。(崇行评:如怕管之假,又安
得有终。)情共日往以月来,谊若天高而地厚。伏愿自盟以后,
相好无尤,更祈人人增有永之年,户户庆无疆之福。凡在时中,
全叨覆庇,谨疏。

政和     年  月  日文疏

\noindent 吴道官读毕,众人拜神已罢,依次又在神前交拜了八拜。(张旁评:
只是如此结拜便了。)然后送神,焚化钱纸,收下福礼去。不一时,吴
道官又早叫人把猪羊卸开,鸡鱼果品之类整理停当,俱是大碗大
盘,摆下两桌。西门庆居于首席,其余依次而坐,吴道官侧席相陪。
须臾,酒过数巡,众人猜枚行令,耍笑哄堂,(张旁评:只是如此便
了。)不必细说。正是:

才见扶桑日出,又看曦驭衔山。

醉后倩人扶去,树稍新月才弯。

饮酒热闹间,只见玳安儿来,附西门庆耳边说道:(张行评:好
收法。)“娘叫小的\banben{接爹来了}[崇本={接爹来去}],说三娘今日发昏哩,请爹早些家去。”
西门庆随即立起来,说道:“不是我摇席破座,委的我第三个小妾十
分病重,咱先去休。”只见花子虚道:“咱与哥同路,咱两个一搭儿去
罢。”伯爵道:“你两个财主的都去了,(崇旁评:口吻极肖。)丢下俺
们怎的?花二哥,你再坐回去。”西门庆道:“他家无人,(张旁评:又
串入瓶儿。)俺两个一搭里去的是,省得他嫂子疑心。”(张行评:意
在斯人,不觉口头溜出,真有此情。)玳安儿道:“小的来时,二娘也
叫天福儿备马来了。”只见一个小厮走近前,向子虚道:“马在这里,
娘请爹家去哩。”于是二人一齐起身,(张行评:独写二人同来同往,【24】
愈衬后文不堪尤甚。)向吴道官致谢打搅,与爵等举手道:“你们
自在耍耍,我们去也。”说着出门上马去了。单留下这几个嚼倒泰山
不谢土的,在庙流连痛饮,不题。

却表西门庆到家,与花子虚别了,进来问吴月娘:“卓二姐怎的
发昏来?”月娘道:“我说一个病人在家,恐怕你搭了这起人,又缠到
那去了,故此叫玳安儿恁地说,(张行评:开手即写月娘无理不通,
真无理不通杀人!天下岂有以他人之死信口出来,作我请人之用
乎?且是对西门庆说,其无礼不通更可恨。)只是一日日觉得重来,
你也要在家看他的是。”西门庆听了,往那边去看,连日在家守着,
不题。(张行评:热结十弟兄已完。)

却说光阴过隙,又早是十月初十外了。(张行评:十月初十外。)
一日,西门庆正使小厮请太医诊视卓二姐病症。刚走到厅上,只见
应伯爵笑嘻嘻走将进来,西门庆与他作了揖,让他坐了。伯爵道:
“哥,嫂子病体如何?”西门庆道:“多分有些不起解,不知怎的好。”
因问:“你们前日多咱时分才散?”伯爵道:“承吴道官再三苦留,散
时也有二更多天气。咱醉的要不的,倒是哥早早来家的便益些。”
(张行评:又足前文。)西门庆因问道:“你吃了饭不曾?”伯爵不好说
不曾吃,因说道:“哥,你试猜。”西门庆道:“你敢是吃了?”伯爵掩口
道:“这等猜不着。”(崇旁评:妙。 张行评:灵极之笔,却为看武松
作势。)西门庆笑道:“怪狗才,不吃便说不曾吃,有这等张致的?”一
面叫小厮:“看饭来,咱与二叔吃。”伯爵笑道:“不然咱也吃了来了,
(张行评:又是这等说入。)咱听得一件稀罕的事儿,来与哥说,要同
哥去瞧瞧。”(张行评:看打虎,前已安线在吴道官口中。今止用伯爵
来说足矣,乃又不肯直出,却闲闲借不吃饭写出。则打虎真是好看,
武松又真是好看。二十分身分,在一闲话描出。《金瓶》笔法惯用此【25】
等也。)西门庆道:“甚么稀罕事?”伯爵道:“就是前日吴道官所说的
景阳岗上那只大虫,昨日被一个人一顿拳头打死了。”西门庆道:
“你又来胡说了,咱不信。”伯爵道:“哥,说也不信。你听着,等我细
说。”于是手舞足蹈说道:(张行评:活现。)“这个人有名有姓,姓武
名松,排行第二,”先前怎的避难在柴大官人庄上,后来怎的害起病
来,病好了又怎的要去寻他哥哥,(张行评:武大郎已出矣。)过这景
阳岗来,怎的遇了这虎,怎的怎的被他一顿拳脚打死了。一五一十
说来,就像是亲见的一般,又像这只猛虎是他打的一般。(张行评:
一段文字,武二出来,武大亦出来,而虚拟打虎、传闻打虎者,色色
皆到,却只是八个“怎的”,两个“像是”便觉奇绝,妙绝。)说毕,西门
庆摇着头儿道:“既恁的,咱与你吃了饭,同去看来。”伯爵道:“哥,
不吃罢,怕误过了。(张行评:又作声价,可知先不吃饭来,非描伯爵
为饭也。)咱们倒不如大街上酒楼上去坐罢。”(张行评:又作卸脱三
人地步。)只见来兴儿放桌儿,西门庆道:“对你娘说,叫别要看饭
了,拿衣服来我穿。”

须臾,换了衣服,与伯爵手拉着手儿同步出来。路上撞着谢希
大,笑道:“哥们敢是来看打虎的么?”(张行评:又作声价。)西门庆
道:“正是。”谢希大道:“大街上好挨挤不开哩。”于是一同到临街一
个大酒楼上坐下。不一时,只听得锣鸣鼓响,众人都一齐瞧看。(张
行评:十倍声价,是好武二。)只见一对对缨枪的猎户,摆将过来,后
面便是那打死的老虎,好像锦布袋一般,四个人还抬不动。(张行
评:是虎。)末后一匹大白马上,坐着一个壮士,就是那打虎的这个
人。(张行评:是打虎者。)西门庆看了,咬着指头道:“你说这等一个
人,若没有千百觔水牛般气力,怎能勾动他一动儿\banben{是的}[崇本={这里}]?”(崇眉评:
伏数语,便挑动酒楼之避,一针不漏。张行评:又照应西门庆这边【26】
一句,又使西门庆心中眼中有一武二也。)三个饮酒评品,按下不
题。(张行评:武二已出,故且用不着药引子也。然而卸脱处又绝不
苟。)

单表迎来的这个壮士怎生模样?但见:

雄躯凛凛,七尺以上身材;阔面稜稜,二十四五年纪。双眸
直竖,远望处犹如两点明星;两手握来,近觑时好似一双铁碓;
脚尖飞起,深山虎豹失精魂;拳手落时,穷谷熊黑皆丧魄。头戴
着一顶万字头巾,上簪两朵银花;身穿着一领血腥衲袄,披着
一方红锦。

\noindent 这人不是别人,就是应伯爵所说阳谷县的武二郎。只为要来寻他哥
子,(张行评:百忙里又点题面,庶下文“冷遇”不突,接笋处不费手
也。)不意中打死了这个猛虎,被知县迎请将来。(张旁评:天下得意
事,都在不意中做出。)众人看着他迎入县里。

\banben{却说这时正值知县升堂}[词本={(无)}],\banben{武松下马进去}[词本={武松到厅上下了轿}],扛着大虫在厅前。知
县看了武松这般模样,心中自忖道:“不恁地,怎打得这个猛虎!”
(张行评:武松又一照。)便唤武松上厅。参见毕,将打虎首尾诉说一
遍。两边官吏都\banben{吓呆}[词本={惊呆}]了。知县\banben{在}[词本={就}]\banben{堂}[崇本={厅}]上赐了\banben{三杯}[词本={几杯}]酒,将库中众土户出
纳的赏钱\banben{五十两}[词本={三十两}],\banben{赐与武松}[词本={就赐与武松}]。武松禀道:“小人托赖相公福阴,偶然
侥幸打死了这个大虫,非小人之能,如何敢受\banben{这些}[词本={这三十两}]赏赐!\banben{众猎户}[词本={给发与众猎户}]因
这畜生,受了相公许多\banben{责罚}[词本={负罚}],\banben{何不就把赏给散与众人}[词本={何不就把这赏给散与众人去}],也显得相公
恩\banben{典}[词本={沾。小人义气},崇本={沾}]。”(崇眉评:不贪财,不伐能,不吝□。张行评:不知者谓是武
松好处,不知此自是作者要武松在清河县中做都头,好遇武大也。)
知县道:“既是如此,任从壮士处分。”武松就把这\banben{五十两}[词本={三十两}]赏钱,在厅
上俵散与众猎户去了。知县见他仁德忠厚,又是一条好汉,有心要
抬举他,便道:“\banben{你虽是阳谷县人氏}[词本={虽是阳谷县的人士}],与我这清河县只在咫尺。我今【27】
日就参你在我县里做个巡捕的都头,\banben{专在}[词本={专一}]河东水西擒拿\banben{贼盗}[词本={盗贼}],你意
下何如?”武松跪谢道:“若蒙恩相抬举小人,终身受赐。”知县随即
唤押司\banben{立了文案}[词本={去了文案}],当日便参武松做了巡捕都头。众里长大户都来与
武松作贺庆喜,\banben{连连吃了数日酒}[词本={连连夸官,吃了三五日酒}]。\banben{正要回阳谷县}[词本={正要阳谷县}]去抓寻哥哥,(张行
评:又入正文。)不料又在清河县做了都头,\banben{却也欢喜}[词本={一日在街上闲游,喜不自胜}]。\banben{那时传得}[词本={传得}]东
平一府两县,皆知武松之名。\banben{正是}[词本={有诗为证}]:

壮士英雄艺略芳,挺身直上景阳岗。

醉来打死山中虎,自此声名播四方。

\banben{却说武松一日在街上闲行,只见背后一个人叫道:“兄弟,(张
行评:二字刺人心肺。)本县相公抬举你做了巡捕都头,怎不看顾
我!”武松回头见了这人,不觉的:欢从额角眉边出,喜逐欢容笑口开。
这人不是别人,却是武松日常间要去寻他的嫡亲哥哥}[词本={按下武松}]武大。(张行
评:方知伯爵口中,及后文两番叙说,为此一句也。)\banben{却说}[词本={单表}]武大自从
兄弟\banben{分别}[词本={分居}]之后,因时遭\banben{饥馑}[词本={荒馑}],搬移在清河县紫石街,赁房居住,人见
他为人懦弱,模样\banben{猥蕤}[词本={猥衰}],起了他个诨名,叫做“三寸丁谷树皮”。(崇
眉评:美名。)俗语言其身上粗糙,头脸窄狭故也。\banben{只因他}[词本={以此人见他}]这般软弱
朴实,多\banben{欺侮}[词本={欺负}]他。\banben{这也不在话下}[词本={武大并无生气,常时回避便了。看官听说,世人惟有人心最歹,软的又欺,恶的又怕,太刚则折,太柔则废。古人有几句格言说的好:柔软立身之本,刚强惹祸之胎。无争无竞是贤才,亏我些儿何碍?青史几场春梦,红尘多才奇才。不须计较巧安排,守分而今见在}]。(张行评:写子虚、武大是一类,是
两样,却不犯手。)

\banben{且说武大无甚生意}[词本={且说武大}],终日挑担子出去街上,卖炊饼度日,不幸
把浑家故了,丢下个女孩儿,年方十二岁,名唤迎儿,爷儿两个过
活。那消半年光景,又消折了资本,移在大街坊张大户家临街房\banben{居
住}[词本={居住,依旧做买卖}]。张宅家下人见他本分,常\banben{照顾}[词本={看顾},崇本={看顾}]他,\banben{照顾他依旧卖些炊饼}[词本={照顾他炊饼}]。闲时
在铺中\banben{坐地}[词本={坐}],武大无不奉承。因此,张宅家人个个都欢喜,在大户面
前,一力与他说方便。因此,大户连房钱也不问武大要。【28】

\banben{却说这张大户}[词本={这张大户家}],有万贯家财,百间\banben{房屋}[词本={房产}],年约六旬之上,身边寸
男尺女皆无。妈妈余氏,主家严厉,房中并无清秀使女。\banben{只因大户
时常拍胸叹气道}[词本={一日,大户拍胸叹了一口气。妈妈问道:“你田产丰盛,资财充足,闲中何故叹气?”大户道}]:“我许大年纪,又无儿女,虽有\banben{几贯家财}[词本={家财}],终何大
用?”妈妈道:“既然如此说,我叫媒人替你买两个使女,早晚习学弹
唱,服侍你便了。”大户\banben{听了}[词本={心中}]大喜,谢了妈妈。过了几时,妈妈果然叫
媒人来,与大户买了两个使女。一个叫做潘金莲,(张行评:出金
莲。)一个唤做白玉莲。\banben{玉莲年方二八,乐户人家出身,生得白净小
巧}[词本={(无)}]。这潘金莲,却是南门外(张行评:南门外,记清。)潘裁的女儿,排
行六姐。因他自幼生得有些\banben{姿色}[词本={颜色}],缠得一双好小脚儿,(崇旁评:是
祸根。)\banben{所以就叫}[词本={因此小名}]金莲。\banben{他父亲}[词本={父亲}]死了,做娘的度日不过,从九岁卖在
王招宣府里(崇旁评:伏脉。张行评:王招宣,须记清。)习学弹唱,
\banben{闲常又教他读书写字。他本性机变伶俐,不过十二三,就会描眉画
眼,傅粉施朱,品竹弹丝,女工针指,知书识字,梳一个缠髻儿,着一
件扣身衫子,做张做致,乔模乔样。(崇旁评:一生伎俩。张行评:
金莲小传,直与西门庆开卷数语相对。)到十五岁的时节}[词本={就会描眉画眼,傅粉施朱,梳一个缠髻儿,着-一件扣身衫子,做张做致,乔模乔样。况他本性机变怜俐,不过十五,就会描鸾刺绣,品竹弹丝,又会一手琵琶}],\banben{王招宣}[词本={后王招宣}]死
了,潘妈妈争将出来,三十两银子转卖与张大户家,与玉莲同时进
门。\banben{大户教他习学弹唱}[词本={大户家习学弹唱}]。\banben{金莲原自会的,甚是省力}[词本={(无)}]。金莲学琵琶,
(张行评:又点琵琶。)\banben{玉莲学筝}[词本={玉莲学筝,玉莲亦年方二八,乃是乐户人家女子,生得白净,小字玉莲}],这两个同房歇卧。主家婆余氏,初
时甚是抬举\banben{二人}[词本={二人,不曾上锅排备洒扫}],与他金银首饰装束身子。后日不料白玉莲死了,
止落下金莲一人,长成一十八岁,出落的脸衬桃花,\banben{眉弯新月}[词本={眉弯新月,尤细尤湾}]。张大
户每要收他,\banben{只碍}[词本={只怕}]主家婆利害,\banben{不得到手}[词本={不得手}]。(崇旁评:倒好。)一日,主
家婆邻家赴席,不在。大户暗把金莲唤至房中,遂收用了。正是:

\banben{莫讶天台相见晚,刘郎还是老刘郎}[词本={美玉无瑕,一朝损坏。珍珠何日,再得完全}]。(崇旁评:趣。)

大户自从收用金莲之后,不觉身上添了四五件病症。(崇旁评:
神效。)端的那五件?(张行评:大户五件病,西门五件事,遥遥相对,【29】
然有事不愁无病也。)第一腰便添疼,第二眼便添泪,第三耳便添
聋,第四鼻便添涕,\banben{第五尿便添滴}[词本={第五尿便添滴。还有一庄儿不可说,白日间只是打盹,到晚来喷睇也无数}]。\banben{自有了这几件病后}[词本={后}],主家婆颇知
其事,与大户嚷骂了数日,将金莲\banben{百般}[词本={甚是}]苦打。\banben{大户知道不容}[词本={大户知不容此女}],却赌气
倒赔房奁,要寻嫁得一个相应的人家。大户家下人都说武大忠厚,
见无妻小,又住着宅内房儿,堪可与他。这大户早晚还要看觑此女,
(崇旁评:有理。)因此不要武大一文钱,白白的嫁与他为妻。这武大
\banben{自从娶了金莲}[词本={自从娶的金莲来家}],大户甚是看顾他。若武大没本钱做炊饼,大户私与
他\banben{银两}[词本={银伍两,与他做本钱}]。武大若挑担儿出去,大户候无人,便踅入房中与金莲厮会。
武大虽一时撞见,\banben{原是他的行货}[词本={(无)}],\banben{不敢}[词本={亦不敢}]声言。朝来暮往,\banben{也有多时}[词本={如此也有(计)〔几〕时}]。
忽一日,大户得患阴寒病症,\banben{呜呼}[词本={呜呼哀哉}]死了。(张行评:金莲起手试手段
处,已斩了一个愚夫。)主家婆察知其事,怒令家僮将金莲、武大即
时赶出。\banben{武大故此遂寻了}[词本={武大不觉,又寻}]紫石街西王皇亲房子,赁内外两间居住,
依旧卖炊饼。

\banben{原来这金莲自嫁武大}[词本={原来金莲自从嫁武大}],见他一味老实,人物猥狼,甚是憎嫌,
(崇旁评:自然。)常与他合气,报怨大户:“普天世界断生了男子,何
故\banben{将我}[词本={将奴}]嫁与这样个货!每日牵着不走,打着\banben{倒退}[词本={倒腿}]的,只是一味味酒,
着紧处,却是\banben{锥钯}[词本={锥扎}]也不动。奴端的那世里悔气,却嫁了他!是好苦
也!”常无人处唱个《山坡羊》为证:

想当初,姻缘错配,奴\banben{把你}[词本={把他}]当男儿汉看觑。不是奴自己夸
奖,他乌鸦怎配鸾凤对?奴真金子埋在土里,他是块高号铜,怎
与俺金色比?他本是块顽石,有甚福抱着我羊脂玉体?好似粪
土上长出灵芝。奈何随他怎样,到底奴心不美。听知奴是块金
砖,怎比泥土基!

看官听说,但凡世上妇女,若自己有些颜色,所禀伶俐,配个好
男子便罢了,若是武大这般,虽好杀,也未免有几分憎嫌。(崇旁评:【30】
况不好乎?)自古佳人才子\banben{相配}[词本={相凑}]着的少。买金偏撞不着卖金的。

武大每日自\banben{挑担儿出去卖炊饼}[词本={挑炊瓶担儿出去}],\banben{到晚方归}[词本={妇人在家别无事干,一日三餐吃了饭,打扮光鲜,只在门前帘儿下站着,常把眉目嘲人,双晴传意。左右街坊有几个奸诈浮浪子弟,睃见了武大这个老婆,打扮油样,沾风惹草。被这干人在街上撒谜语,往来嘲戏,唱叫:这一块好羊肉,如何落在狗口里!人人自知武大是个懦弱之人,却不知他娶得这个婆娘在屋里,风流伶俐,诸般都好,为头的一件,好偷汉子。有诗为证:金莲容貌更堪题,笑蹙春山八字眉。若遇风流清子弟,等闲云雨便偷期}]。\banben{那妇人}[词本={这妇人}]每日打发武
大出门,只在帘子下嗑瓜子儿,(崇旁评:好消遣。张行评:此处已
伏帘子。)一径把那一对小金莲\banben{故露}[词本={做露}]出来,\banben{勾引浮浪子弟}[词本={勾引的这伙人}],日逐在门
前弹胡博词,\banben{撒谜语}[词本={扠儿难}],\banben{叫唱:“一块好羊肉,如何落在狗口里}[词本={(无)}]?”油似
滑的言语,无般不说出来。因此,武大在紫石街又住不牢,要往别处
搬移,与老婆商议。妇人道:“贼馄饨,不晓事的,你赁人家房住,浅
房浅屋,可知有小人啰唣!不如凑几两银子,看相应的,典上他两间
住,却也气概些,免受人欺侮。”\banben{武大道:“我那里有钱典房?”妇人
道:“呸,浊才料!你是个男子汉,倒摆布不开,常交老娘受气}[词本={你是个男子汉,倒摆布不开,常交老娘受气。武大道:“我那里有钱典房?”妇人道:“呸,浊才料}]!\banben{没有
银子}[词本={(无)}],把我的钗梳凑办了去,有何难处!过后有了,再治不迟。”(崇
眉评:此处亦复能贤。张行评:本来犹可为善,则王婆可剐也。)武
大听老婆这般说,当下凑了十数两银子,典了县门前楼上下两层四
间房屋居住。第二层是楼,两个小小院落,甚是干净。武大自从搬
到县西街上来,\banben{照旧卖炊饼过活}[词本={照旧卖炊饼}]。(张行评:此一篇清析文字,下文
用“不想这日”四字,便瞒过插入的这一篇文字去。妙,妙!)

\banben{不想这日,撞见自己嫡亲兄弟}[词本={一日,街上所过,见数队缨枪,锣鼓喧天,花红软轿,簇拥着一个人,却是他嫡亲兄弟武松。因在景阳冈打死了大虫,知县相公抬举他,新升做了巡捕都头,街上里老人等作贺他,送他下处去。却被武大撞见,一手扯住,叫道:兄弟,你今日做了都头,怎不看顾我!武松回头见是哥哥。}]。\banben{当日兄弟相见}[词本={二人相合}],心中大喜。一
面邀请到家中,让至楼上坐,房里唤出金莲来,与武松相见,因说
道:“前日景阳岗打死了大虫的,便是你小叔。(崇旁评:好不气概。)
今新充了都头,是我一母同胞兄弟。”(崇旁评:值得卖弄。 张行
评:遥映“热结”。)那妇人\banben{叉手便向前}[词本={叉手向前}],便道:“叔叔(张行评:一。)万
福。”武松施礼,倒身下拜。妇人扶住武松道:“叔叔(张行评:二。)请
起,折杀奴家。”武松道:“嫂嫂受礼。”两个相让了一回,都平磕了头
起来。少顷,小女迎儿拿茶,二人吃了。武松见妇人十分妖娆,只把
头来低着。(崇旁评:不老气。 张行评:写妇人,写武松,毛发皆【31】
动。)不多时,武大安排酒饭,\banben{款待}[词本={管待}]武松。

说话中间,武大下楼买酒菜去了,丢下妇人,独自在楼上陪武
松坐地。看了武松身材凛凛,相貌堂堂,(崇眉评:此想入神。)\banben{又想
他打死了那大虫,毕竟有千百觔气力}[词本={身上恰似有千百斤气力,不然如何打得那大虫}],(崇旁评:慧想,慧想!)\banben{口中
不说}[词本={(无)}],\banben{心下思量道}[词本={心里寻思道}]:(张行评:又从打虎上入妇人心事,我固云《金
瓶》惯用此曲笔也。)“\banben{一母所生的兄弟}[词本={一母所生的兄弟,又这般长大,人物壮键,奴若嫁得这个,胡乱也罢了}],\banben{怎生}[词本={你看}]我家那身不满尺的丁
树,三分似人,七分似鬼,奴那世里遭瘟,\banben{撞着他来}[词本={直到如今}]?\banben{如今看起武松
这般人物壮健}[词本={据看武松又好气力}],何不叫他搬来我家住?\banben{想这段姻缘却在这里了}[词本={谁想这段缘却在这里}]。”
(崇旁评:且看。)\banben{于是一面堆下笑来}[词本={那妇人一面脸上排下笑来}],问道:“叔叔,(张行评:三。)你
如今在那里居住?每日饭食谁人整理?”武松道:“武二新充了都头,
逐日答应上司,别处住不方便,胡乱在县前寻了个下处,每日拨两
个土兵\banben{伏侍}[词本={服事}]做饭。”妇人道:“叔叔(张行评:四。)何不搬来家里住?
省的在县前土兵\banben{服侍}[词本={服事}]做饭腌臜。一家里住,早晚要些汤水吃时,也
方便些。就是奴家亲自安排与叔叔(张行评:五。)吃,也干净。”武松
道:“深谢嫂嫂。”妇人又道:“莫不别处有婶婶?(崇旁评:细心。)可
请来厮会。”武松道:“武二并不曾婚娶。”妇人道:“叔叔(张行评:
六。)青春多少?”武松道:“虚度二十八岁。”妇人道:“原来叔叔(张
行评:七。)倒长奴三岁。叔叔(张行评:八。)今番从那里来?”武松
道:“在沧州住了一年有余。只想哥哥在旧房居住,\banben{不道移在这里}[词本={不想搬在这里}]。”
妇人道:“一言难尽。自从嫁得你哥哥,吃他忒善了,被人欺负,\banben{才到
这里来}[词本={才得到这里}]。若是叔叔(张行评:九。)这般雄壮,(崇旁评:二字得心应
口。)谁敢道个不字!”武松道:“家兄从来本分,不似武松撒泼。”(崇
旁评:和盘托出。)妇人笑道:“怎的颠倒说!常言‘人无刚强,安身不
长’。奴家平生\banben{性快}[词本={快性}],\banben{看不上那三打不回头,四打和身转的}[词本={看不上这样三打不回头、四打和身转的人。有诗为证,诗曰:叔嫂萍踪得偶逢,娇娆偏逞秀仪容。私心便欲成欢会,暗把邪言钓武松。原来这妇人甚是言语撇清}]。”武松
道:“家兄不惹祸,免得嫂嫂忧心。”(张行评:一路纯是白描勾挑。)【32】
\banben{二人在楼上一递一句的说}[词本={二人只在楼上}]。\banben{有诗为证:叔嫂萍踪得偶逢,娇娆偏逞秀仪容。私心便欲成欢会,暗把邪言钓武松。}[词本={(无)}]

\banben{话说金莲陪着武松在楼上}[词本={(无)}],说话未了,只见武大买了些肉菜果
饼\banben{归家}[词本={归来}],放在厨下,走上楼来,\banben{问道}[词本={叫道}]:“大嫂,你且下来\banben{则个}[词本={安排则个}]。”那妇人
应道:“你看那不晓事的,叔叔(张行评:十。)在此,无人陪侍,却交
我撇了下去。”(崇旁评:哥哥也陪得,不必定要嫂嫂。)武松道:“嫂
嫂请方便。”妇人道:“何不去间壁\banben{请王干娘来安排}[词本={请王干娘来安排便了}]?(崇旁评:伏
脉。)只是这般不见便。”(张行评:又出王婆。)武大便自去央了间壁
王婆来,安排端正,都拿上楼来,摆在桌子上,无非是些鱼肉果菜点
心之类。随即盪上酒来。武大叫妇人坐了主位,武松对席,武大打
横。三人坐下,把酒来斟,武大筛酒在各人面前。那妇人拿起酒来,
道:“叔叔(张行评:十一。)休怪,没甚管待,请杯儿水酒。”武松道:
“感谢嫂嫂,休这般说。”武大只顾上下筛酒,那妇人笑容可掬,满口
儿叫:“叔叔,(张行评:将上文无数叔叔,至此一总。)怎的肉果儿也
不拣一箸儿?”(崇旁评:还有肉卷儿哩。)拣好的递将过来。武松是
个直性的汉子,只把做亲嫂嫂相待——谁知这妇人是个使女出身,
惯会小意儿——\banben{亦不想这妇人一片引人心}[词本={亦不想这妇人一片引人心。那武大又是善弱的人,那里会管待人}]。那妇人陪武松吃了几
杯酒,一双眼只看着武松身上。武松\banben{吃他}[词本={乞他}]看不过,\banben{只得倒低了头}[词本={只低了头,不理他}]。
(崇旁评:二官太嫩。张行评:又描妇人、武二一遍。)吃了一歇,酒
阑了,便起身。武大道:“二哥没事,再吃几杯儿去。”武松道:“生受。
我再来望哥哥、嫂嫂罢。”都送下楼来。出的门外,妇人便道:“叔叔,
(张行评:十二。此又一叔叔也。)\banben{你是必上心}[词本={是必上心},崇本={是必上心}]搬来家里住,若是不搬
来,俺两口儿也吃别人笑话,亲兄弟难比别人,(张行评:虽是金莲
的话,却是一回的总结,试思文不一总,只顾写下半回,如何结上半【33】
回?文字照顾之法,全在人不能测也。)与我们争口气,也是好处。”
(崇旁评:大义激之。)武松道:“既是\banben{嫂嫂}[词本={吾嫂}]厚意,今晚有行李便取
来。”妇人道:“\banben{奴这里等候哩}[词本={叔叔是必记心者,奴这里专候}]!”(张行评:悠然而结。)正是:

满前野意无人识,\banben{几点碧桃春自开}[词本={几点碧桃春自开。有诗为证:可怪金莲用意深,包藏淫行荡春心。武松正大原难犯,耿耿清名抵万金。当日这妇人情意十分殷勤}]。【34】

\printyiwenlist
\end{diben}

\end{document}
